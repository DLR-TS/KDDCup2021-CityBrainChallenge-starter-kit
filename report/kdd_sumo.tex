%%
%% This is file `sample-authordraft.tex',
%% generated with the docstrip utility.
%%
%% The original source files were:
%%
%% samples.dtx  (with options: `authordraft')
%% 
%% IMPORTANT NOTICE:
%% 
%% For the copyright see the source file.
%% 
%% Any modified versions of this file must be renamed
%% with new filenames distinct from sample-authordraft.tex.
%% 
%% For distribution of the original source see the terms
%% for copying and modification in the file samples.dtx.
%% 
%% This generated file may be distributed as long as the
%% original source files, as listed above, are part of the
%% same distribution. (The sources need not necessarily be
%% in the same archive or directory.)
%%
%% The first command in your LaTeX source must be the \documentclass command.
\documentclass[sigconf,authordraft]{acmart}
%% NOTE that a single column version may required for 
%% submission and peer review. This can be done by changing
%% the \doucmentclass[...]{acmart} in this template to 
%% \documentclass[manuscript,screen]{acmart}
%% 
%% To ensure 100% compatibility, please check the white list of
%% approved LaTeX packages to be used with the Master Article Template at
%% https://www.acm.org/publications/taps/whitelist-of-latex-packages 
%% before creating your document. The white list page provides 
%% information on how to submit additional LaTeX packages for 
%% review and adoption.
%% Fonts used in the template cannot be substituted; margin 
%% adjustments are not allowed.

%%
%% \BibTeX command to typeset BibTeX logo in the docs
\AtBeginDocument{%
  \providecommand\BibTeX{{%
    \normalfont B\kern-0.5em{\scshape i\kern-0.25em b}\kern-0.8em\TeX}}}

%% Rights management information.  This information is sent to you
%% when you complete the rights form.  These commands have SAMPLE
%% values in them; it is your responsibility as an author to replace
%% the commands and values with those provided to you when you
%% complete the rights form.
\setcopyright{acmcopyright}
\copyrightyear{2021}
\acmYear{2021}
\acmDOI{10.1145/1122445.1122456}

%% These commands are for a PROCEEDINGS abstract or paper.
\acmConference{City Brain Challenge, KDD Cup 2021}{April 01--June 05, 2021}{Online}
\acmBooktitle{}
\acmPrice{}
\acmISBN{}


%%
%% end of the preamble, start of the body of the document source.
\begin{document}

%%
%% The "title" command has an optional parameter,
%% allowing the author to define a "short title" to be used in page headers.
\title[SUMO Lane Queue Agent]{The Traffic Light Lane Queue Agent of Team SUMO}

%%
%% The "author" command and its associated commands are used to define
%% the authors and their affiliations.
%% Of note is the shared affiliation of the first two authors, and the
%% "authornote" and "authornotemark" commands
%% used to denote shared contribution to the research.
\author{Robert Alms}
\affiliation{%
  \institution{German Aerospace Center, Institute of Transportation Systems}
  \streetaddress{Rutherfordstr. 2}
  \city{Berlin}
  \country{Germany}}
\email{robert.alms@dlr.de}

\author{Michael Behrisch}
\affiliation{%
  \institution{German Aerospace Center, Institute of Transportation Systems}
  \streetaddress{Rutherfordstr. 2}
  \city{Berlin}
  \country{Germany}}
\email{michael.behrisch@dlr.de}

\author{Jakob Erdmann}
\affiliation{%
  \institution{German Aerospace Center, Institute of Transportation Systems}
  \streetaddress{Rutherfordstr. 2}
  \city{Berlin}
  \country{Germany}}
\email{jakob.erdmann@dlr.de}

\author{Yun-Pang Fl\"otter\"od}
\affiliation{%
  \institution{German Aerospace Center, Institute of Transportation Systems}
  \streetaddress{Rutherfordstr. 2}
  \city{Berlin}
  \country{Germany}}
\email{yun-pang.floetteroed@dlr.de}

\author{Peter Wagner}
\affiliation{%
  \institution{German Aerospace Center, Institute of Transportation Systems}
  \streetaddress{Rutherfordstr. 2}
  \city{Berlin}
  \country{Germany}}
\email{peter.wagner@dlr.de}

%%
%% By default, the full list of authors will be used in the page
%% headers. Often, this list is too long, and will overlap
%% other information printed in the page headers. This command allows
%% the author to define a more concise list
%% of authors' names for this purpose.
\renewcommand{\shortauthors}{Alms and Behrisch, et al.}

%%
%% The abstract is a short summary of the work to be presented in the
%% article.
\begin{abstract}
  A simple rule based algorithm is described which optimizes a single intersection
  and is tested and parameter-optimized against a whole network.
  The algorithm observes/calculates the queue-lengths upstream
  of the intersection and also possible congestions downstream.
  It favors phases where two lanes which currently have demand
  get green to maximize the outflow.
\end{abstract}

%%
%% The code below is generated by the tool at http://dl.acm.org/ccs.cfm.
%% Please copy and paste the code instead of the example below.
%%
\begin{CCSXML}
<ccs2012>
<concept>
<concept_id>10010405.10010432.10010439</concept_id>
<concept_desc>Applied computing~Engineering</concept_desc>
<concept_significance>500</concept_significance>
</concept>
</ccs2012>
\end{CCSXML}

\ccsdesc[500]{Applied computing~Engineering}

%%
%% Keywords. The author(s) should pick words that accurately describe
%% the work being presented. Separate the keywords with commas.
\keywords{traffic simulation, traffic lights, optimization}

%%
%% This command processes the author and affiliation and title
%% information and builds the first part of the formatted document.
\maketitle

\section{Introduction}
The City Brain Challenge 2021 provided a platform to test traffic light control 
algorithms a against a real city network. The simulation scenario consists of 
the street network together with an exemplary traffic demand and the simulation tool itself
provided in a docker container. Every 10 seconds the algorithms under test were provided
speed and location of every vehicle currently in the network and could switch 
the traffic light phase at every junction accordingly. While the network was known to 
the algorithm neither the testing demand nor the source code of the simulator
were available. The input demand in the final phase covered only 20 minutes.

The primary idea of the challenge was to test machine learning algorithms in
the traffic light domain probably by setting up a simple algorithm which can then
be parameterized individually for every junction possibly even over time.

Our team however took a different approach which was to check how far we can get
with a general purpose algorithm which works essentially the same on all junctions
and maybe also estimate the effect of the objective function on the whole system.
In the following a description of the implemented algorithm is given together with 
a very rough adhoc estimation on the usefulness of the different improvements
ending with some remarks on the general simulation setup and the objective function.

Remark: The name of the team stems from a simulation suite \emph{Eclipse SUMO} which is being
developed mainly at our institute and supported a lot of our analysis.

\section{Description of the algorithm}

The code can be divided into a pre-processing step and the runtime algorithm
where the pre-processing is nothing more than an analysis of the test route set
for the expected remaining travel time.

By evaluating the given 
\texttt{flow\_round3\_flow0.txt} as well as two additional demand files generated 
by the provided \texttt{traffic\_generator.py} an average value for the remaining 
free flow travel time for vehicles travelling a certain edge can be calculated.
This value is of course only valid for
the given demand but since it depends on the structure of the network and it is
to be expected that also the final demand will be generated with a script similar
to the \texttt{traffic\_generator.py}, the data has been used for the final demand as well.

The basic structure of the run-time algorithm has four steps:
\begin{enumerate}
    \item aggregating vehicle information
    \item evaluating lane queues
    \item choosing the best phase
    \item post-processing to ensure a maximum green time and remove long term jams
\end{enumerate}

\subsection{Vehicle Information}
A self-made observation function has been implemented which basically forwards
the complete results of \texttt{self.eng.get\_lane\_vehicles()} to the agent
which is a dictionary mapping each lane of a traffic light (an agent) to the list
of vehicles on that lane.

The agent itself now extracts the needed information for the vehicle. First of all
it tries to rebuild the historic route of the vehicle by finding a shortest path between
the previously observed and the current position of the vehicle. This way the driven part
of the routes of all vehicles currently in the scenario is known. If the vehicle route
ended already a special marker is appended to the stored route.

Furthermore the free flow travel time for the driven route is calculated and stored
for the vehicle.

After all initial vehicle info has been evaluated one can calculate for each
edge in the scenario a distribution of possible future routes such that one 
gets a (continuously updated) probability distribution for the future route of each vehicle.

\subsection{Lane Queues}
The second step is to calculate the lane queue lengths for every incoming lane at a junction.
The basic idea is that the priority that a lane gets green should depend on the
number of waiting vehicles on that lane. But not all vehicles on the lane are
\emph{equal} in the sense that they have different chances to pass the junction
(either because of their distance to the junction or because of their destination
lane being jammed) in the next interval but also different expected contributions
to the objective function (see the section on the objective function). As a result 
the vehicles are assigned different weights.

While calculating the lane queues at a junction there is repeatedly the need to determine
whether the destination lane(s) for an incoming lane can still accept vehicles or is jammed.
So the algorithm calculates for every outgoing lane whether it either has many vehicles
and a low average speed or a lot of vehicles in the insertion buffer (a region at the begin
of the lane where vehicles are placed which want to enter the lane but have not found 
a suitable position yet). Those lanes are marked as jammed.

Now the vehicles which contribute to the queue are determined by checking their time distance
(speed over distance) to the stop line against a threshold and respecting (almost) standing vehicles.
For those vehicles the probability that their destination lane is jammed is calculated. If all destination
lanes are jammed the probability is $1$, if all are free it is $0$. For all the remaining cases the 
possible future routes are evaluated from the probability distribution determined earlier.

Summing over all possible future routes it is possible to determine a probability that the target lane 
of the vehicle is jammed. In case of an unknown future lane a calibratable constant is used. This
gives for each vehicle the probability $p_j$to have a jammed lane ahead. The second important
factor is the (expected) free flow travel time $t_f$ of the vehicle (see the section on the objective 
function for a detailed discussion). Since an expected value for the remaining time has been determined
by the pre-processing and the free flow time up to the current edge is known precisely.

The weight of a vehicle can now be determined using the following equation:
\begin{equation}
w = \frac{T}{t_f} (1- p_j)
\end{equation}
where $T$ is the average expected free flow travel time over all vehicles (again a constant 
which can be calibrated). Please observe that the value may range from $0$ to $\infty$ but 
a \emph{regular} vehicle should have a weight of about $1$.

The weight of a lane queue is now simply the sum over all weights of the vehicles considered relevant.
In order to cover the case of short edges leading to a junction where upstream jams may extend to
different edges this evaluation is also extended to all predecessor lanes on upstream roads but only if
there is no traffic light junction inbetween.

\subsection{Long term jams}
For every lane which has more than a defined number of vehicles on it and operates below
a defined speed threshold a jam bonus is collected over time leading to eventually 
resolved long term jams. This jam bonus is added to every lane right at the end of the queue
length calculation. Unlike the queue length above which is recalculated on every iteration,
this bonus adds up over time until a switch to a different traffic light phase occurs.

\section{Improvements}
The basic algorithm is now described but as can be seen it involves several thresholds and constants
which need to be adapted. In total the algorithm has 23 parameters which were optimized indivdually.
In order to do so first the possible ranges for all parameters are determined and then a rastering
of the value range has been done by using equidistant samples (usually ten) in the range of values.
This way near optimal values wirth respect to the test data set could be reached.

\begin{table*}
  \caption{Parameters of the model}
  \label{tab:param}
  \begin{tabular}{lrlc}
    \toprule
    Name&Final value&Description&Range\\
    \midrule
DUAL\_SWITCH\_THRESH& 0.1 & difference in queue lengths which triggers switch optimized &-1 19\\
SWITCH\_THRESH & 0.5 & difference in queue lengths which triggers switch optimized &-0.1 0.9\\
ROUTE\_LENGTH\_WEIGHT & 689.0 & to be optimized &600 1200\\
MIN\_CHECK\_LENGTH & 29.0 & look upstream to find more queued vehicles if a lane is shorter than this &25 65\\
JAM\_THRESH & 0 & at which relative occupancy a lane is considered jammed&0 2\\
SPEED\_THRESH & 0.025 & at which relative speed a lane is considered jammed&0 0.1\\
JAM\_BONUS & 0.1 & bonus vehicles to add to a jammed lane per act call (every 10s) until it gets green & 0 1\\
HEADWAY & 2.05 & to be optimized& 1.2 2.2\\
SLOW\_THRESH & 0 & at which relative speed a vehicle is considered slow, to be optimized &0.05 0.15\\
MAX\_GREEN\_SEC & 164.0 & to be optimized &140 220\\
PREFER\_DUAL\_THRESHOLD & 4.0 & How many vehicles could pass the intersection in 5s all red &1 11\\
& & (if we have more vehicles in a dual queue switching, beats single queue discharge)\\
STOP\_LINE\_HEADWAY & 10.8 & seconds to the stopline to be included in the queue &8 12\\
BUFFER\_THRESH & 1.0 & number of vehicles in lane insertion buffer to consider lane jammed &1 11\\
FUTURE\_JAM\_LOOKAHEAD & 3 & number of edges & 3 13\\
SATURATED\_THRESHOLD & 44 & optimized &30 50\\
SATURATION\_INC & 5& optimized &5 15\\
MIN\_ROUTE\_PROB & 0.9 & optimized &0 1\\
MIN\_ROUTE\_COUNT & 7.0 & optimized &0 20\\
DELAY\_WEIGHT & 35.0 & optimized &0 50\\
LEFT\_TURN\_BONUS & 30 & optimized &0 50\\
BUFFER\_JAM\_THRESH & 1.8 &&\\
BUFFER\_SPEED\_THRESH & 0 &&\\
UNKNOWN\_LANE\_JAM\_PROB & 1.2 & optimized &0.6 1.6\\
  \bottomrule
\end{tabular}
\end{table*}


In the process several automated multidimensional optimization schemes like \texttt{optimparallel},
\texttt{scikit-optimize} or \texttt{scipy.optimize.fmin\_cobyla} have been tested as well but due to the
very fragile environment (high random fluctuation of the results) they were considered too unstable.

\section{Objective Function}

\section{Evaluation using SUMO}
In order to evaluate the 


\section{Outlook}
All the resulting code will be made publicly available on GitHub once all publications in connection 
with the algorithm are done.

\begin{acks}
Thank you to the organizers of the competition.
\end{acks}


\end{document}
%%
%% End of file `sample-authordraft.tex'.
